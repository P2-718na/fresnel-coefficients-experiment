\subsection{Calcolo incertezza strumentale}\label{subsec:calcolo-incertezza-strumentale}
  Per effettuare le misure d'intensità luminosa, Arduino compara
  il segnale $V_{sig}$ in uscita dal sensore \emph{TEMT6000} con il potenziale
  $V_{cc}$ dell'alimentatore.
  Dato che il segnale $V_{cc}$ può essere soggetto a fluttuazioni sistematiche significative\footnote{Basti pensare a un alimentatore che si riscalda durante la presa dati,
  causando un progressivo abbassamento di $V_{cc}$.}
  (dell'ordine del ~10\%)\footnote{Arduino può essere alimentato con potenziale compreso tra $\sim 4.5V$ e $\sim 5.5V$.}, abbiamo utilizzato il potenziale di riferimento
  $V_{ref}$ di \emph{Arduino}, mantenuto costante a $1.1V$, come strumento di
  calibrazione.
  Per ogni misurazione l'\emph{ADC} svolge un calcolo di questo tipo:
  %
  \begin{equation}
    V_i' = k * \frac {V_i} {V_{cc}}
    \label{eq:misura-adc}
  \end{equation}
  %
  \noindent dove $V_i'$ è il valore di misurato, $V_i$ è il valore vero e $k$
  è una costante di proporzionalità. Conoscendo il valore nominale
  di $V_{ref}$\footnote{Trascuriamo il valore iniziale di \emph{offset}, visto che
  a noi importa solo della precisione della misura e non della sua accuratezza.}, possiamo rimuovere la
  dipendenza da $V_{cc}$. Per ogni misura $I_i$ di intensità, andiamo a misurare sia $V_{sig}$ che $V_{ref}$.
  I valori $V_{sig}'$ e $V_{ref}'$ che otteniamo sono dati dalla \eqref{eq:misura-adc}:
  %
  \vspace{-10mm}
  \begin{multicols}{2}
    \begin{equation}
      V_{ref}' = k * \frac {V_{ref}} {V_{cc}}
      \label{eq:vref-misura}
    \end{equation}
  \break
    \begin{equation}
      V_{sig}' = k * \frac {V_{sig}} {V_{cc}}
      \label{eq:vsig-misura}
    \end{equation}
  \end{multicols}
  %
  \noindent Dividendo la \eqref{eq:vsig-misura} per la \eqref{eq:vref-misura} troviamo il valore vero cercato $V_{sig}$.
  La formula finale è quindi (riportata con le incertezze già calcolate):
  %
  \begin{equation}
    V_{sig} = \frac {
      V_{ref} (\pm 0.4\%)
    } {
      V'_{ref} (\pm 0.2\%)
    } V'_{sig} (\pm 0.2\%)
    \label{eq:misura-intensità}
  \end{equation}
  %
  In questa formula compaiono due diverse sorgenti di errore:
  l'\emph{ADC} di \emph{ATmega328P} e il potenziale di riferimento $V_{ref}$
  di Arduino.
  % Questo magari va messo in una nota?
  %abbiamo considerato trascurabile l’incertezza strumentale del sensore, dato
  %che non è riportata nel \emph{datasheet} e
  %qualsiasi suo effetto sarà incluso
  %nelle incertezze casuali stimate in (vedi sezione 6.2).
  Abbiamo ottenuto l’incertezza dell’\emph{ADC} direttamente dal
  \emph{datasheet}\cite{atmega} di \emph{ATmega328P} (Sez.23.1):
  il produttore indica un'incertezza massima assoluta di $\pm 2LSB \approx 0.2\%$.
  Questo valore tiene conto di tutte le possibili fonti di errore interne
  all’\emph{ADC} ed è da considerarsi come limite superiore.
  Per quanto riguarda l'incertezza del potenziale di riferimento, abbiamo
  utilizzato il lavoro di Main\cite{main}, dove è stato misurato un errore sistematico
  massimo dello $0.4\%$.
  Propagando linearmente le incertezze nella formula \eqref{eq:misura-intensità} si ottiene un incertezza
  strumentale dello $0.8\%$.

\subsection{Calcolo incertezza casuale}\label{subsec:calcolo-incertezza-casuale}
  % Questa incertezza è data dall'allineamento del sensore.
  A causa di limitazioni strumentali, non ci è stato possibile ottenere un campione
  numeroso di misure per ogni angolo. Possiamo ottenere ugualmente
  una stima delle incertezze casuali assumendo che:
  1) le misure di intensità sono distribuite in modo normale attorno a un valore
    medio $\mu_i$, proprio di ogni angolo;
  2) tra un angolo e un altro la distribuzione differisce solo per la media e non per la
    deviazione standard $\sigma$.
  Queste sono giustificate dal fatto che il modo in cui vengono raccolti i dati
  non è dipendente dall'angolo a cui si trova il sensore.
  Abbiamo quindi stimato $\sigma$ tenendo in considerazione degli scarti quadratici
  dei campioni di ogni angolo.
  Abbiamo usato la formula:
  %
  \begin{equation}
    \sigma = \sqrt{
      \frac {
        \sum_{i = \theta_{min}}^{\theta_{max}} \left(
          \sum_{j = 0}^{N_i} (x_{ij} -\mu_i)^2
        \right)
      } {
        \nu
      }
    }
  \end{equation}
  %
  \noindent dove $N_i$ è il numero di dati nel campione $\vec{x}_i$.
  $\nu$ è il numero di gradi di libertà dei dati. La miglior stima per l'incertezza
  sull'intensità luminosa, infine, è data dalla variazione standard della media $\sigma^*$.
  Il risultato ottenuto è $\sigma^* = 7.30 (\text{unità arbitrarie})$.
                                                                      % noi interessano le fluttuazioni
                                                                      % statistiche, quindi questa roba non viene
                                                                      % divisa per sqrt(n). Se ci interessasse la
                                                                      % singola misura, dovremmo
                                                                      % dividere.
  % btw qui ci sarebbe da dividere per sqrt(2), visto che ho preso 2 punti per ogni angolo.
\endinput

% Continuo con la roba dell'errore. Arduino può misurare un errore di intensità
% per via di Vcc. Ora, vcc può avere variazioni rapide, ma può anche avere
% variaziuoni più lente, dell'ordine di tutto lo sweep di angoli, che sono
% in pratica erorri sistematici. Questo fixa tutto.

\subsection{Calcolo incertezza strumentale}\label{subsec:calcolo-incertezza-strumentale}
  Per effettuare le misure d'intensità luminosa, la scheda \emph{Arduino} compara
  il segnale $V_{sig}$ in uscita dal sensore \emph{TEMT6000} con il potenziale
  $V_{cc}$ dell'alimentatore.
  Dato che il segnale $V_{cc}$ può essere soggetto a fluttuazioni significative % todo precisa che le fluttuazioni avvengono in tempi maggiori a quelli del ciclo di clock di arduino
  (dell'ordine del ~10\%), abbiamo utilizzato il potenziale di riferimento
  $V_{ref}$ di \emph{Arduino}, mantenuto costante a $1.1V$, come strumento di
  calibrazione.
  Sappiamo che per ogni misurazione l'\emph{ADC} svolge un calcolo di questo tipo:

  \begin{equation}
    V_i' = k * \frac {V_i} {V_{cc}}
  \end{equation}

  \noindent dove $V_i'$ è il valore di misurato, $V_i$ è il valore vero e $k$
  è una costante di proporzionalità. Visto che conosciamo il valore nominale
  di $V_{ref}$, possiamo sfruttare queste informazioni per rimuovere la
  dipendenza da $V_{cc}$. Per ogni misura $I_i$ di intensità, misuriamo sia
  il potenziale del sensore $V_{sig}$ che il potenziale di riferimento $V_{ref}$.
  I valori $V_{sig}'$ e $V_{ref}'$ che otterremo saranno dati da (??): %todo add ref

  \vspace{-10mm}
  \begin{multicols}{2}
    \begin{equation}
      V_{ref}' = k * \frac {V_{ref}} {V_{cc}}
    \end{equation}
  \break
    \begin{equation}
      V_{sig}' = k * \frac {V_{sig}} {V_{cc}}
    \end{equation}
  \end{multicols}

  \noindent Dividento la (ref) per la (ref) troviamo il valore vero cercato $V_{sig}$.%todo add ref
  La formula finale è quindi (riportata con gli errori già calcolati):

  % fixme questa formula è da sistemare. Il fatto è che in realtà v_sig non è
  %  espresso in volt, ma in LSB. (Intrinsecamente, dentro di lui compare di
  %  nuovo V_cc, quindi i due si semplificano.
  \begin{equation}
    V_{sig} = \frac {
      V_{ref} (\pm 0.4\%)
    } {
      V'_{ref} (\pm 0.2\%)
    } V'_{sig} (\pm 0.2\%)
    \label{eq:misura-intensità}
  \end{equation}

  \noindent In realtà nella nostra implementazione il valore di $V_{sig}$ è
  moltiplicato per una costante $k = (1024 \cdot 5)^{-1 V^{-1}}$, in modo
  da ottenere un intervallo di valori in uscita compreso in un range tra $0$ e
  $1000$. % fixme ok questo continua ad essere sus. Però almeno non influenza l'errore yee

  In questa formula compaiono quindi due diverse sorgenti di errore:
  l'\emph{ADC} di \emph{ATmega328P} e il potenziale di riferimento $V_{ref}$
  di \emph{Arduino Uno}.
  % Questo magari va messo in una nota?
  %abbiamo considerato trascurabile l’incertezza strumentale del sensore, dato
  %che non è riportata nel \emph{datasheet} e %todo add reference to datasheet
  %qualsiasi suo effetto sarà incluso
  %nelle incertezze casuali stimate in (vedi sezione 6.2). %todo add reference to section
  Abbiamo ottenuto l’incertezza dell’\emph{ADC} direttamente dal
  \emph{datasheet} di \emph{ATmega328P} (sezione 23.1 ADC  Features): % todo add reference
  il produttore indica un'incertezza massima assoluta di $\pm 2LSB \approx 0.2\%$. % todo add note for LSB
  Questo valore tiene conto di tutte le possibili fonti di errore interne
  all’\emph{ADC} ed è da considerarsi come limite superiore.
  Per quanto riguarda l'incertezza del potenziale di riferimento, abbiamo
  utilizzato il lavoro di Main[ref], dove è stato misurato un errore %todo add reference
  massimo dello $0.4\%$.
  Propagando linearmente le incertezze nella formula (??) si ottiene un incertezza
  strumentale dello $0.8\%$ nelle misure di intensità.

\subsection{Calcolo incertezza casuale}\label{subsec:calcolo-incertezza-casuale}
  % Questa incertezza è data dall'allineamento del sensore.
  A causa di limitazioni strumentali, non ci è stato possibile ottenere un campione
  numeroso di misure per ogni angolo. Possiamo ottenere ugualmente
  una stima delle incertezze casuali facendo le assunzioni: % fixme no fare pls
  - Le misure di intensità sono distribuite in modo normale attorno a un valore
    medio $\mu_i$, proprio di ogni angolo.
  - Tra un angolo e un altro la distribuzione differisce solo per la media e non per la
    deviazione standard $\sigma$.
  Queste sono giustificate dal fatto che il modo in cui vengono raccolti i dati
  non è dipendente dall'angolo a cui si trova il sensore.
  Di conseguenza, possiamo ottenere una stima di $\sigma$ tenendo in considerazione
  gli scarti quadratici dei campioni di ogni angolo, unitamente.
  Abbiamo usato la formula:

  \begin{equation}
    \sigma = \sqrt{
      \frac {
        \sum_{i = \theta_{min}}^{\theta_{max}} \left(
          \sum_{j = 0}^{N_i} (x_{ij} -\mu_i)^2
        \right)
      } {
        \nu
      }
    }
  \end{equation}

  \noindent dove $N_i$ è il numero di dati nel campione $\vec{x}_i$.
  $\nu$ è il numero di gradi di libertà dei dati, calcolato come
  $(n^\circ \text{ dati raccolti}) - (n^\circ \text{ parametri calcolati dai dati})$.
  Il risultato ottenuto è $\sigma = 9.46 (\text{unità arbitrarie})$.  % noi interessano le fluttuazioni
                                                                      % statistiche, quindi questa roba non viene
                                                                      % divisa per sqrt(n). Se ci interessasse la
                                                                      % singola misura, dovremmo
                                                                      % dividere.
  % btw qui ci sarebbe da dividere per sqrt(2), visto che ho preso 2 punti per ogni angolo.
\endinput

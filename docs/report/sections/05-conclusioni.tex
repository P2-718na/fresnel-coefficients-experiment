Per quanto riguarda la prima parte dell'esperimento, i dati raccolti seguono l'andamento
previsto. Il \emph{fit} ottenuto è soddisfacente, e
il valore ottenuto per $n_2$ è compatibile con il coefficiente di rifrazione del
vetro $n = 1.5$.
Il valore di chi-quadro
$\tilde \chi^2_\pi < 1$ indica che c'è stata una sovrastima delle incertezze, che può
essere dovuta al numero ridotto di misture effettuate.
Riguardo alla seconda parte dell'esperimento, abbiamo ottenuto due stime di
$\theta_B$ in parziale disaccordo: $\theta_B = 56.2^\circ \pm 0.1^\circ$ e $\theta_B = 56.8^\circ \pm 0.2^\circ$.
Tuttavia, l'indice di rifrazione tipico del vetro è compreso tra $n=1.50$ e $n=1.53$, e se
inseriamo questi valori nell'equazione \eqref{eq:legge-brewster} osserviamo che i risultati
ottenuti rientrano nell'intervallo di valori attesi $56.3^\circ < \theta_{B} < 56.8^\circ$.

Nello svolgere l'esperimento siamo stati fortemente limitati dal tempo a disposizione
e siamo confidenti che il nostro apparato sperimentale possa ottenere risultati ancora
più accurati se potessimo acquisire ulteriori set di dati. In ogni caso, consideriamo
questo esperimento un successo.
\endinput

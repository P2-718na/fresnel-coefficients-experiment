\subsection{Analisi preliminare, verifica delle leggi di Fresnel}\label{subsec:analisi-dati}
  In questo modo, abbiamo ottenuto le misure di intensità riflessa $I_\pi$ e $I_\sigma$ [fig ??]. Per ottenere i
  coefficienti di Fresnel, abbiamo normalizzato i dati ottenuti tenendo conto che:
  there is no difference between Rp and Rs at normal incidence
  at glancing angle in the less-dense medium the reflection coefficients are ±1,
  Come descritto in [Lipson]
  Da queste misure abbiamo ricavato anche l’angolo di Brewster, osservando per quale valore di $\theta_i$
  l’intensità $I_\pi$ si annulla.
  %TODO
\subsection{Misura dell'angolo di Brewster}\label{subsec:angolo-di-brewster}
  \blindtext[2]
  % TODO
\endinput

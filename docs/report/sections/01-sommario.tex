In questo esperimento abbiamo analizzato il comportamento di un fascio laser polarizzato
che colpisce la superficie di un prisma di vetro.
In particolare, abbiamo verificato che l’intensità del fascio laser riflesso segua l’andamento
previsto dalle equazioni di Fresnel e abbiamo misurato l'angolo di Brewster per il materiale del prisma.
Per la prima parte dell'esperimento, abbiamo svolto un \emph{fit} sui dati d'intensità luminosa raccolti; i valori di chi-quadro ridotto
ottenuti sono di $\tilde \chi^2_\pi = 0.71$ per la luce \pi-polarizzata e
$\tilde \chi^2_\sigma = 1.17$ per la luce \sigma-polarizzata, a cui corrispondono le probabilità $P(\tilde \chi^2 < \tilde \chi^2_\pi) = 0.10$ e $P(\tilde \chi^2 < \tilde \chi^2_\sigma) = 0.82$.
Per la seconda parte abbiamo usato due metodi diversi per trovare l'angolo di Brewster.
I risultati ottenuti $\theta_{B1} = 56.2^\circ \pm 0.1^\circ$ e $\theta_{B2} = 56.8^\circ \pm 0.2^\circ$ sono in parziale disaccordo tra di loro,
ma sono compatibili con l'intervallo di valori atteso.
\endinput

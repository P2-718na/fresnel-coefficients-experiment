\documentclass[12pt, a4paper, twoside]{article}

\usepackage{fontspec}
\usepackage{blindtext}
\usepackage{geometry}
\usepackage{setspace}
\usepackage{titlesec}
\usepackage{indentfirst}
\usepackage{graphicx}
\usepackage[italian]{babel}
\usepackage{catchfile}
\usepackage{multicol}
\usepackage{amsmath}
\usepackage{subcaption}

% FIXME swap to calibri on windows? no license for linux
\setmainfont{Carlito}
\titlespacing*{\section}{0px}{3mm}{1mm}
\titlespacing*{\subsection}{0px}{3mm}{1mm}
\geometry{
  left=2cm,
  right=2cm,
  top=2cm,
  bottom=2cm
}
\setlength{\parindent}{10mm}
\graphicspath{ {./assets}, {../../assets} }

% Allow use of command \getenv{VARNAME}.
% Taken from: https://tex.stackexchange.com/questions/62010/can-i-access-system-environment-variables-from-latex-for-instance-home
\newcommand{\getenv}[2][]{
  \CatchFileEdef{\temp}{"|kpsewhich --var-value #2"}{\endlinechar=-1}%
  \if\relax\detokenize{#1}\relax\temp\else\let#1\temp\fi}

\begin{document}

\begin{center}
  \huge Misura dell'angolo di Brewster di un prisma mediante le formule di Fresnel \\ %todo this is just a placeholder title for now
  \large Francesco Platini \getenv{MAT1}. Matteo Bonacini \getenv{MAT2}.\\ % see readme on how to use this
  \today
\end{center}

% Title spacing

\section{Sommario}\label{sec:abstract}
  In questo esperimento abbiamo analizzato il comportamento di un raggio di luce
  che colpisce la superficie di un prisma di materiale ignoto.
  Abbiamo verificato che l’intensità luminosa riflessa segue l’andamento
  previsto dalle leggi di Fresnel e abbiamo utilizzato queste leggi per misurare
  il valore dell'angolo di Brewster per il prisma.
  I risultati ottenuti sono ?????????? compatibili/incompatibili % todo

\section{Introduzione}\label{sec:introduzione}
  Quando un raggio luminoso colpisce una superficie di separazione tra due
  dielettrici con indici di rifrazione $n_1$ e $n_2$ diversi, si possono verificare
  i fenomeni della riflessione e rifrazione della luce.
  Le equazioni di Fresnel, derivabili direttamente dalle equazioni di Maxwell, % fixme in equazioni ci va maiuscola?
  descrivono questo comportamento prevedendo quale frazione d'intensità
  luminosa viene riflessa e quale rifratta, per ciascuna delle due componenti di
  polarizzazione.
  Indichiamo con $R_\pi$ e $R_\sigma$ le frazioni d'intensità luminosa riflessa
  polarizzate, rispettivamente, parallelamente e perpendicolarmente al piano
  d'incidenza.
  Si dimostra che valgono:

  \begin{equation}
    R_\pi = \frac {
      n_2 \cos{\theta_i} - n_1 \cos{\theta_t}
    } {
      n_2” \cos{\theta_i} + n_1 \cos{\theta_t}
    }\label{eq:fresnel equation 1}
  \end{equation}

  \begin{equation}
    R_\sigma = \frac {
      n_1 \cos{\theta_i} - n_2 \cos{\theta_t}
    } {
      n_1” \cos{\theta_i} + n_2 \cos{\theta_t}
    }\label{eq:fresnel equation 2}
  \end{equation}

  \noindent dove $n_1$ e $n_2$ sono gli indici di rifrazione dei due mezzi, $\theta_i$ e
  $\theta_r$ sono gli angoli d'incidenza e rifrazione del raggio luminoso.
  Una dimostrazione rigorosa di come si ricavino queste equazioni non è oggetto
  di questo testo, ma si rimanda a Mazzoldi[1] per ulteriori approfondimenti. % todo add reference
  Variando l’angolo d'incidenza e misurando l’intensità in funzione di tale, si
  può vedere l’andamento dei due coefficienti.
  Le formule prevedono che $R_\pi$ si annulli per un certo valore di $\theta_i$,
  che indicheremo con $\theta_B$.
  Questo angolo prende il nome di Angolo di Brewster, ed è dato dalla Legge di Brewster:

  \begin{equation}
    \theta_B = \arctan{
      \frac {n_2} {n_1}as
    }\label{eq:Brewster's law}
  \end{equation}

 \noindent  L’angolo di Brewster gode di alcune proprietà utili in diverse applicazioni
  sperimentali.
  Per approfondimenti, rimandiamo a [Mazzoldi].



\section{Apparato sperimentale e svolgimento}\label{sec:apparato-sperimentale-e-svolgimento}
  \subsection{Apparato sperimentale}\label{subsec:apparato-sperimentale}
    L’apparato sperimentale, riportato in [fig.1] è formato da un prisma
    posizionato al centro di una guida circolare graduata. Il prisma è montato
    ad un servomotore (??? modello) che gli permette di ruotare con una
    risoluzione angolare di 1.0° +/- 0.5°. (todo: il goniometro ha permesso di
    misurare l’angolo iniziale da cui far partire l’acquisizione). Un laser è
    puntato verso il cristallo, e due filtri polarizzatori sono posti in mezzo
    al fascio. Alla guida rotante è collegato un sensore di intensità luminosa
    (??? Modello).
    Il servo ed il sensore sono collegati a (?? mettere davvero tutto il setup
    oppure solamente una roba) una scheda arduino, che si occupa dell’acquisizione
    dei dati.
    (todo inserisci numeri della roba)

    Todo add specifiche di tutta la roba e link to datasheet.
    ADC arduino, reference voltage e cazzate varie
    Considerazioni su come l’errore sul sensore sia determinato principalmente da difficoltà nell’allineamento e da fluttuazioni del laser.

    \begin{figure}[h]
      \centering
      \caption{Apparato sperimentale e schema circuitale.}
      \begin{subfigure}{.4\textwidth}
        \includegraphics[width=7cm]{instrumental-apparatus.png}
        \caption{
          \emph{
            Apparato sperimentale. A partire da sinistra, in senso orario,
            si trovano: laser(1), filtri polaroid(2, 3), guida circolare(4),
            prisma(5), sensore(6), Arduino(7). Il servomotore non è riportato.
          }
        }
        \label{fig:instrumental-apparatus}
      \end{subfigure}%
      \hspace{20mm}
      \begin{subfigure}{.4\textwidth}
        \includegraphics[width=7cm]{circuit-diagram.png}
        \caption{
          \emph{
            Schema circuitale. I pin $VCC$ e $GND$ del sensore sono collegati
            all'alimentazione di Arduino. Il pin $SIG$
            del sensore è collegato ad un input analogico di Arduino, tramite una
            resistenza da $10k\Omega$.
          }
        }
        \label{fig:circuit-diagram}
      \end{subfigure}
    \end{figure}

  \subsection{Procedura sperimentale}\label{subsec:procedura-sperimentale}
    Abbiamo iniziato misurando l’andamento del coefficiente $R_\pi$. Partiamo posizionando il secondo polaroid in modo che la polarizzazione della luce sia parallela al lato del prisma. Il laser che abbiamo utilizzato emetteva luce polarizzata, quindi abbiamo dovuto aggiustare anche il suo angolo rispetto al polaroid. Prima di iniziare le misure, abbiamo regolato l’intensità del laser in modo che il sensore potesse rilevare un range di valori il più ampio possibile. Abbiamo quindi posizionato il sensore direttamente davanti al laser e abbiamo ruotato il primo filtro polaroid, fino a che il sensore non è riuscito a rilevare una variazione significativa di intensità.
    Per prendere le misure, abbiamo iniziato ruotando il prisma in modo da massimizzare $\theta_i$. Per raccogliere un
    punto dati abbiamo allineato il sensore con il fascio laser riflesso e ne abbiamo misurato l’intensità. Abbiamo
    ripetuto questa misura riducendo man mano il valore di $\theta_i$, fino ad arrivare il più vicino possibile a $0\deg$.
    Il nostro apparato sperimentale ci ha permesso di svolgere misure per ?? < $\theta_i$ < ??.
    Abbiamo poi ruotato il secondo polaroid di 90° e ripetuta la stessa procedura per misurare $R_\sigma$.
    In questo modo, abbiamo ottenuto le misure di intensità riflessa $I_\pi$ e $I_\sigma$ [fig ??]. Per ottenere i
    coefficienti di Fresnel, abbiamo normalizzato i dati ottenuti tenendo conto che:
    there is no difference between Rp and Rs at normal incidence
    at glancing angle in the less-dense medium the reflection coefficients are ±1,
    Come descritto in [Lipson]
    Da queste misure abbiamo ricavato anche l’angolo di Brewster, osservando per quale valore di $\theta_i$
    l’intensità $I_\pi$ si annulla.


\section{Risultati e discussione}\label{sec:risultati-e-discussione}
  \subsection{Verifica delle leggi di Fresnel}\label{subsec:leggi-di-fresnel}
    \blindtext[2]
  \subsection{Misura dell'angolo di Brewster}\label{subsec:angolo-di-brewster}
    \blindtext[2]

\section{Conclusioni}\label{sec:conclusioni}
  \blindtext[2]

\newpage
\section{Appendice}\label{sec:appendice}
  \subsection{Calcolo incertezza strumentale}\label{subsec:calcolo-incertezza-strumentale}
    Per effettuare le misure d'intensità luminosa, la scheda \emph{Arduino} compara
    il segnale $V_{sig}$ in uscita dal sensore \emph{TEMT6000} con il potenziale
    $V_{cc}$ dell'alimentatore.
    Dato che il segnale $V_{cc}$ può essere soggetto a fluttuazioni significative % todo precisa che le fluttuazioni avvengono in tempi maggiori a quelli del ciclo di clock di arduino
    (dell'ordine del ~10\%), abbiamo utilizzato il potenziale di riferimento
    $V_{ref}$ di \emph{Arduino}, mantenuto costante a $1.1V$, come strumento di
    calibrazione.
    Sappiamo che per ogni misurazione l'\emph{ADC} svolge un calcolo di questo tipo:

    \begin{equation}
      V_i' = k * \frac {V_i} {V_{cc}}
    \end{equation}

    \noindent dove $V_i'$ è il valore di misurato, $V_i$ è il valore vero e $k$
    è una costante di proporzionalità. Visto che conosciamo il valore nominale
    di $V_{ref}$, possiamo sfruttare queste informazioni per rimuovere la
    dipendenza da $V_{cc}$. Per ogni misura $I_i$ di intensità, misuriamo sia
    il potenziale del sensore $V_{sig}$ che il potenziale di riferimento $V_{ref}$.
    I valori $V_{sig}'$ e $V_{ref}'$ che otterremo saranno dati da (??): %todo add ref

    \vspace{-10mm}
    \begin{multicols}{2}
      \begin{equation}
        V_{ref}' = k * \frac {V_{ref}} {V_{cc}}
      \end{equation}
    \break
      \begin{equation}
        V_{sig}' = k * \frac {V_{sig}} {V_{cc}}
      \end{equation}
    \end{multicols}

    \noindent Dividento la (ref) per la (ref) troviamo il valore vero cercato $V_{sig}$.%todo add ref
    La formula finale è quindi (riportata con gli errori già calcolati):

    % fixme questa formula è da sistemare. Il fatto è che in realtà v_sig non è
    %  espresso in volt, ma in LSB. (Intrinsecamente, dentro di lui compare di
    %  nuovo V_cc, quindi i due si semplificano.
    \begin{equation}
      V_{sig} = \frac {
        V_{ref} (\pm 0.4\%)
      } {
        V'_{ref} (\pm 0.2\%)
      } V'_{sig} (\pm 0.2\%)
      \label{eq:misura-intensità}
    \end{equation}

    \noindent In realtà nella nostra implementazione il valore di $V_{sig}$ è
    moltiplicato per una costante $k = (1024 \cdot 5)^{-1 V^{-1}}$, in modo
    da ottenere un intervallo di valori in uscita compreso in un range tra $0$ e
    $1000$. % fixme ok questo continua ad essere sus. Però almeno non influenza l'errore yee

    In questa formula compaiono quindi due diverse sorgenti di errore:
    l'\emph{ADC} di \emph{ATmega328P} e il potenziale di riferimento $V_{ref}$
    di \emph{Arduino Uno}.
    % Questo magari va messo in una nota?
    %abbiamo considerato trascurabile l’incertezza strumentale del sensore, dato
    %che non è riportata nel \emph{datasheet} e %todo add reference to datasheet
    %qualsiasi suo effetto sarà incluso
    %nelle incertezze casuali stimate in (vedi sezione 6.2). %todo add reference to section
    Abbiamo ottenuto l’incertezza dell’\emph{ADC} direttamente dal
    \emph{datasheet} di \emph{ATmega328P} (sezione 23.1 ADC  Features): % todo add reference
    il produttore indica un'incertezza massima assoluta di $\pm 2LSB \approx 0.2\%$. % todo add note for LSB
    Questo valore tiene conto di tutte le possibili fonti di errore interne
    all’\emph{ADC} ed è da considerarsi come limite superiore.
    Per quanto riguarda l'incertezza del potenziale di riferimento, abbiamo
    utilizzato il lavoro di Main[ref], dove è stato misurato un errore %todo add reference
    massimo dello $0.4\%$.
    Propagando linearmente le incertezze nella formula (??) si ottiene un incertezza
    strumentale dello $0.8\%$ nelle misure di intensità.

  \subsection{Calcolo incertezza casuale}\label{subsec:calcolo-incertezza-casuale}
    A causa di limitazioni strumentali, non ci è stato possibile ottenere un campione
    numeroso di misure per ogni angolo. Possiamo ottenere ugualmente
    una stima delle incertezze casuali facendo le assunzioni: % fixme no fare pls
    - Le misure di intensità sono distribuite in modo normale attorno a un valore
      medio $\mu_i$, proprio di ogni angolo.
    - Tra un angolo e un altro la distribuzione differisce solo per la media e non per la
      deviazione standard $\sigma$.
    Queste sono giustificate dal fatto che il modo in cui vengono raccolti i dati
    non è dipendente dall'angolo a cui si trova il sensore.
    Di conseguenza, possiamo ottenere una stima di $\sigma$ tenendo in considerazione
    gli scarti quadratici dei campioni di ogni angolo, unitamente.
    Abbiamo usato la formula:

    \begin{equation}
      \sigma = \sqrt{
        \frac {
          \sum_{i = \theta_{min}}^{\theta_{max}} \left(
            \sum_{j = 0}^{N_i} (x_{ij} -\mu_i)^2
          \right)
        } {
          \nu
        }
      }
    \end{equation}

    \noindent dove $N_i$ è il numero di dati nel campione $\vec{x}_i$.
    $\nu$ è il numero di gradi di libertà dei dati, calcolato come
    $(n^\circ \text{ dati raccolti}) - (n^\circ \text{ parametri calcolati dai dati})$.
    Il risultato ottenuto è $\sigma = 9.46 (\text{unità arbitrarie})$.  % noi interessano le fluttuazioni
                                                                        % statistiche, quindi questa roba non viene
                                                                        % divisa per sqrt(n). Se ci interessasse la
                                                                        % singola misura, dovremmo
                                                                        % dividere.
    % btw qui ci sarebbe da dividere per sqrt(2), visto che ho preso 2 punti per ogni angolo.
    \cite{mazzoldi98}
  \subsection{References}
    \bibliographystyle{plain} % We choose the "plain" reference style
    \bibliography{./docs/report/fresnelReportRefs.bib} % Entries are in the fresnelReportRefs.bib file
\end{document}

%In this document somasdmeters
%were added. There is an encoding package,
%and pagesize and fontsize parameters.
